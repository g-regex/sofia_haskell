\documentclass[notitlepage]{report}
%\listfiles
\usepackage{mystyle}
\raggedbottom
\def\footnoterule{\vfill % added this
   \kern-3pt\hrule width 2truein \kern 2.6pt}

\usepackage{titling}
\usepackage{standalone}
\usepackage[scaled=0.95]{DejaVuSans}
\renewcommand*\familydefault{\sfdefault}

%\usepackage[T1]{fontenc}
\usepackage[utf8]{inputenc}
%\usepackage[scaled=0.85]{DejaVuSansMono}
\usepackage[scaled=0.95]{DejaVuSansMono}

\usepackage[fencedCode,inlineFootnotes,citations,definitionLists,
            hashEnumerators,smartEllipses,pipeTables,tableCaptions,
            hybrid]{markdown}

%\usepackage{showframe}
\usepackage{etoolbox}
\makeatletter
\patchcmd{\@makechapterhead}{\vspace*{50\p@}}{}{}{}
\patchcmd{\@makeschapterhead}{\vspace*{50\p@}}{}{}{}

\newcommand\m[1]{\mintinline{haskell}{#1}}
\newcommand\mh[1]{\mintinline{html}{#1}}
\newcommand\ms[1]{\mintinline{sql}{#1}}

\usepackage{pmboxdraw}

\DeclareUnicodeCharacter{255A}{\pmboxdrawuni{255A}}
\DeclareUnicodeCharacter{2554}{\pmboxdrawuni{2554}}
\DeclareUnicodeCharacter{2551}{\pmboxdrawuni{2551}}

\pretitle{\begin{center}\Huge\bfseries}
\posttitle{\par\end{center}}
\preauthor{\begin{center}\Large\ttfamily}
\postauthor{\end{center}}
\predate{\par\large\centering}
\postdate{\par}

\title{SOFiA\\
\large A proof assistant implementation in Haskell}
\author{Gregor Feierabend}
\date{\today}

\begin{document}
%\begin{titlepage}
%\begin{haddocktitle}
%Sofia/Haskell documentation
%\end{haddocktitle}
%\end{titlepage}

\maketitle

\begingroup
\let\clearpage\relax
\tableofcontents
\endgroup

\chapter{Introduction}
The purpose of this report is to document the work on the Haskell
implementation of the SOFiA proof assistant, which is based on the earlier
Python implementation.\footnote{See: %
\url{https://github.com/ZurabJanelidze/sofia}}
The document is work in progress and targets contributors to the project. Hence
no further introduction to the context of the project is given. In the
description of the Python implementation the name of this software is motivated
as follows.
\begin{quote}
SOFiA stands for "Synaptic First-Order Assembler". The small "i" indicates that
the default base of the language is a bit less than that of intuitionistic
logic. Namely, disjunction and false are not implemented.
\end{quote}
Note that the given Haskell code  makes use of the \m{Parser} data type, created
by \textcite{Hutton}, which is available on the web. The rest of the source code
and this documentation stand under the GNU General Public License
v3.0\footnote{See: \url{https://www.gnu.org/licenses/}}. The source code of this
software is hosted on GitHub\footnote{See:
\url{https://github.com/g-regex/sofia_haskell}}. During the current stage of
development, however, the repository is private.

\markdownInput{../README.md}

\section{Grammar}
We agreed on the following EBNF\footnote{Extended Backus-Naur Form} of the
grammar of the Sofia language.
\begin{figure}[h!]
    {\renewcommand{\arraystretch}{2.0}
    \documentclass{standalone}

\usepackage[utf8]{inputenc}
\usepackage{mathtools} 
\usepackage{booktabs}

\begin{document}

\newcommand{\expression}{\langle\texttt{expression}\rangle}
\newcommand{\statement}{\langle\texttt{statement}\rangle}
\newcommand{\formula}{\langle\texttt{formula}\rangle}
\newcommand{\formulator}{\langle\texttt{formulator}\rangle}
\newcommand{\character}{\langle\texttt{character}\rangle}
\newcommand{\atom}{\langle\texttt{atom}\rangle}
\newcommand{\symb}{\langle\texttt{symbol}\rangle}

\begin{tabular}{lll}
$\expression$   & $\coloneqq$ &	$\formula \Big| \statement$                   \\
$\formula$      & $\coloneqq$ &	$\formulator \bigg[ \statement \Big\{
                                 \formulator \statement \Big\} \Big[
                                 \formulator \Big] \bigg]$                    \\
                &             &	$\Big| \statement \formulator \Big\{ \statement
                                 \formulator \Big\} \Big[ \statement \Big]$   \\
$\statement$    & $\coloneqq$ & $\atom \Big\{ \atom \Big\}$                   \\
$\atom$		    & $\coloneqq$ & $\textrm{`['} \formula   \textrm{`]'} \Big|
                                 \textrm{`['} \statement \textrm{`]'} \Big|
                                 \textrm{`['} \textrm{`]'}$                   \\
$\formulator$   & $\coloneqq$ & $\textrm{`:'} \Big| \textrm{`='} \Big| \symb$ \\
$\symb$         & $\coloneqq$ & $\character \Big\{ \character \Big\}$         \\
$\character$	& $\coloneqq$ & $\textrm{`a'} \Big| \dots \Big| \textrm{`z'}
                                 \Big| \textrm{`A'} \Big| \dots \Big|
                                 \textrm{`Z'} \Big| \textrm{`0'} \Big| \dots
                                 \Big| \textrm{`9'} \Big| \textrm{`+'} \Big|
                                 \textrm{`-'} \Big| \textrm{`/'}
                                 \Big| \textrm{`*'} \Big|\textrm{`\%'} \Big|
                                 \textrm{`\textasciicircum'} \Big| \textrm{`\&'}
                                 \Big| \textrm{`.'} \Big|\textrm{`?'} \Big|
                                 \textrm{`!'} \Big| \textrm{`\_'}$
\end{tabular}

\end{document}
}
    \caption{EBNF of the Sofia language}\label{fig:EBNFsofia}
\end{figure}

Detailed information about the representation of formal languages by means of an
EBNF can be found online\footnote{%
See: \url{https://en.wikipedia.org/wiki/EBNF}}. For convenience,
the meaning and/or terminology of the symbols used above is given below.
\begin{multicols}{2}
\begin{itemize}
    \item $\langle\texttt{name}\rangle$ is called a non-terminal or a production
    \item $(\cdots)$ groups elements
    \item $|$ means `or'
    \item $\{\cdots\}$ means zero or more repetitions
    \item $[\cdots]$ means zero or one occurrences.
\end{itemize}
\end{multicols}

We will not discuss the various parsers used in this project in detail, but only
mention the EBNFs of the relevant grammars and refer to the relevant module
which implements the parser. The parser for the grammar described in
Figure~\ref{fig:EBNFsofia} is implemented in the \m{SofiaParser} module.

\chapter{Significant Data Structures}
\input{haddock/SofiaTree}

\chapter{Function Reference Sofia}
\input{haddock/Sofia}

\chapter{User Interface}
The web interface through which the user interacts with Sofia provides an input
field imitating a terminal command line. For the sake of uniformity the proof
building command to be entered in this interface resemble the Haskell functions
of the \m{Sofia} module - except for some small adjustments and the fact that no
partial proof has to be given as the last argument.
\\\\
The EBNF for these commands looks as follows:
\begin{figure}[h!]
    {\renewcommand{\arraystretch}{2.0}
    \documentclass{standalone}

\usepackage[utf8]{inputenc}
\usepackage{mathtools} 
\usepackage{booktabs}

\usepackage[scaled=0.95]{DejaVuSans}
\renewcommand*\familydefault{\sfdefault}

\usepackage[scaled=0.95]{DejaVuSansMono}

\renewcommand\baselinestretch{1.2}

\begin{document}

\newcommand{\command}{\langle\texttt{command}\rangle}
\newcommand{\assume}{\langle\texttt{assume}\rangle}
\newcommand{\restate}{\langle\texttt{restate}\rangle}
\newcommand{\synapsis}{\langle\texttt{synapsis}\rangle}
\newcommand{\apply}{\langle\texttt{apply}\rangle}
\newcommand{\rightsub}{\langle\texttt{rightsub}\rangle}
\newcommand{\leftsub}{\langle\texttt{leftsub}\rangle}
\newcommand{\recall}{\langle\texttt{recall}\rangle}
\newcommand{\intgr}{\langle\texttt{int}\rangle}
\newcommand{\pair}{\langle\texttt{pair}\rangle}
\newcommand{\pairlist}{\langle\texttt{pairlist}\rangle}
\newcommand{\intlist}{\langle\texttt{intlist}\rangle}
\newcommand{\strlist}{\langle\texttt{strlist}\rangle}
\newcommand{\strg}{\langle\texttt{string}\rangle}
\newcommand{\character}{\langle\texttt{character}\rangle}

\begin{tabular}{lll}
$\command$      & $\coloneqq$ &	$\assume \Big| \restate \Big| \recall \Big|
                                 \synapsis \Big| \apply \Big| \rightsub \Big|
                                 \leftsub$                                    \\
$\assume$       & $\coloneqq$ &	$\textsf{`assume'}␣\strg$                     \\
$\recall$       & $\coloneqq$ &	$\textsf{`recall'}␣\intgr␣\strlist$           \\
$\restate$      & $\coloneqq$ &	$\textsf{`restate'}␣\pairlist␣\strg$          \\
$\synapsis$     & $\coloneqq$ &	$\textsf{`synapsis'}$                         \\
$\apply$        & $\coloneqq$ &	$\textsf{`apply'}␣\intgr␣\pairlist␣\intgr$    \\
$\rightsub$     & $\coloneqq$ &	$\textsf{`rightsub'}␣\intgr␣\intgr␣\intlist␣
                                 \intgr␣\intgr$                               \\
$\leftsub$      & $\coloneqq$ &	$\textsf{`leftsub'}␣\intgr␣\intgr␣\intlist␣
                                 \intgr␣\intgr$                               \\
$\pair$         & $\coloneqq$ & $\textsf{`('} \intgr \textsf{`,'} \intgr
                                 \textsf{`)'}$                                \\
$\pairlist$     & $\coloneqq$ & $\textsf{`[]'}\Big|\textsf{`['} \pair
                                 \Big\{\textsf{`,'} \pair
                                 \Big\} \textsf{`]'}$                         \\
$\intlist$      & $\coloneqq$ & $\textsf{`[]'}\Big|\textsf{`['} \intgr
                                 \Big\{\textsf{`,'} \intgr
                                 \Big\} \textsf{`]'}$                         \\
$\strlist$      & $\coloneqq$ & $\textsf{`[]'}\Big|\textsf{`['} \strg
                                 \Big\{\textsf{`,'} \strg
                                 \Big\} \textsf{`]'}$                         \\
$\intgr$      	& $\coloneqq$ & $\Big( \textsf{`0'} \Big| \dots \Big|
                                 \textsf{`9'} \Big)
                                 \Big\{
                                 \textsf{`0'} \Big| \dots \Big| \textsf{`9'}
                                 \Big\}$                                      \\
$\strg$         & $\coloneqq$ & $\textsf{`"'} \Big\{ \character \Big\}
                                 \textsf{`"'}$                                \\
$\character$	& $\coloneqq$ & $\textsf{`0'} \Big| \dots \Big| \textsf{`9'}
                                 \Big| \textsf{`A'} \Big| \dots \Big|
                                 \textsf{`Z'} \Big| \textsf{`0'} \Big| \dots
                                 \Big| \textsf{`9'} \Big| \textsf{`+'} \Big|
                                 \textsf{`-'} \Big| \textsf{`/'}
                                 \Big| \textsf{`*'} \Big|\textsf{`\%'} \Big|
                                 \textsf{`\textasciicircum'} \Big| \textsf{`\&'}
                                 \Big| \textsf{`.'} \Big|\textsf{`?'} \Big|
                                 \textsf{`!'} \Big| \textsf{`\_'}$
\end{tabular}

\end{document}
}
    \caption{EBNF for the commands of the Sofia
             user interface}\label{fig:EBNFcmds}
\end{figure}

When constructing a proof, the command and the history of all commands used to
build up the currently visible proof are sent via \texttt{POST} to the backend.
The backend then checks whether all supplied arguments are valid and calculates
the proof that results from executing all the Haskell functions corresponding to
the commands in the history and the current command in sequence.
\\
There are also some `meta commands', which do not operate on the current proof
but are rather concerned with the appearance of the user interface, storing the
current proof as a postulate or loading a proof from the database. Parsing of
all commands is accomplished by the \m{SofiaCommandParser} module.

\section{Database}
To store axioms, axiom builders, propositions and proofs (for the purposes of
the database they are all the same), we use a table in an SQLite database. This
table has the following form:

\begin{minted}{sql}
CREATE TABLE "axiom_builder"(
    "id"        INTEGER PRIMARY KEY,
    "rubric"    VARCHAR NOT NULL,
    "name"      VARCHAR NOT NULL,
    "schema"    VARCHAR NOT NULL,
    "params"    INTEGER NOT NULL,
    "desc"      VARCHAR NOT NULL,
    "hist"      VARCHAR NOT NULL,
    CONSTRAINT "axiom" UNIQUE ("rubric","name"))
\end{minted}

Each entry as an auto-incrementing (default setting) \ms{"id"} as the primary
key. Further, each entry can be identified by a \emph{unique} combination of a
\ms{"rubric"} and a \ms{"name"} (e.g.\ \texttt{Arithmetic} and
\texttt{Induction}). The column \ms{"schema"} contains string representations of
\m{AxiomSchema}ta (see Section~\ref{sec:axiom_schemata}). The column
\ms{"params"} contains the number of parameters required by the \m{AxiomSchema}
- this value is determined by the \m{AxiomSchema}, but is stored for convenience
in the table as well. The \ms{"desc"} column contains some string containing a
human readable description of the axiom builder for the user. Finally, the
\ms{"hist"} column may contain a history of commands that where used to obtain a
postulate (or example proof, theorem, etc.); in the case of axioms the field
contains the empty string.\\\\
Using the \texttt{Persistent} library\footnote{A helpful introduction to
\texttt{Persistent} can be found here:
\url{https://www.yesodweb.com/book/persistent}}, a Haskell data type is created
which directly corresponds to this table:

\newcommand{\htt}[1]{{\small\bfseries\texttt{#1}}}

\begin{haddockdesc}
\item[\mintinline{haskell}{
data AxiomBuilder
}]
{\haddockbegindoc
\enspace \emph{Constructors}\par
\begin{tabulary}{\linewidth}{@{}llJ@{}}
    \htt{=}  &   \htt{AxiomBuilder}         &                 \\
             &   \htt{\{  axiomBuilderRubric } &\htt{:: !String} \\
             &   \htt{,   axiomBuilderName   } &\htt{:: !String} \\
             &   \htt{,   axiomBuilderSchema } &\htt{:: !String} \\
             &   \htt{,   axiomBuilderParams } &\htt{:: !Int   } \\
             &   \htt{,   axiomBuilderDesc   } &\htt{:: !String} \\
             &   \htt{,   axiomBuilderHist   } &\htt{:: !String \}}
 \end{tabulary}}
\end{haddockdesc}

From within the handler functions requests to the database backend (SQLite) are
directly converted to an \m{Entity}\footnote{More information about this data
type can be found in the \texttt{Persistent} documentation.} of \m{AxiomBuilder},
which can then conveniently be accessed and processed further by other Haskell
functions.

\pagebreak
\section{Axiom schemata}\label{sec:axiom_schemata}
\input{haddock/SofiaAxiomParser}
The syntax for encoding \m{AxiomSchema}ta as strings is defined by the following
EBNF:
\begin{figure}[h!]
    {\renewcommand{\arraystretch}{2.0}
    \documentclass{standalone}

\usepackage[utf8]{inputenc}
\usepackage{mathtools} 
\usepackage{booktabs}

\usepackage[scaled=0.95]{DejaVuSans}
\renewcommand*\familydefault{\sfdefault}

\usepackage[scaled=0.95]{DejaVuSansMono}

\renewcommand\baselinestretch{1.2}

\begin{document}

\newcommand{\axiom}{\langle\texttt{axiom}\rangle}
\newcommand{\builder}{\langle\texttt{builder}\rangle}
\newcommand{\leaf}{\langle\texttt{leaf}\rangle}
\newcommand{\lst}{\langle\texttt{list}\rangle}
\newcommand{\itm}{\langle\texttt{item}\rangle}
\newcommand{\intgr}{\langle\texttt{int}\rangle}
\newcommand{\strg}{\langle\texttt{string}\rangle}
\newcommand{\character}{\langle\texttt{character}\rangle}

\begin{tabular}{lll}
$\axiom$        & $\coloneqq$ &	$\Big( \textsf{`\{'} \strg \textsf{`,'} \intgr
                                 \textsf{`,'} \lst \textsf{`\}'}\Big) \Big|
                                 \Big( \textsf{`\{'} \intgr \textsf{`,'} \intgr
                                 \textsf{`,'} \itm \textsf{`\}'}\Big)$        \\
$\lst $         & $\coloneqq$ &	$\textsf{`['} \itm \Big\{\textsf{`,'}
                                 \itm \Big\}\textsf{`]'}$                     \\
$\itm $         & $\coloneqq$ &	$\intgr \Big| \strg \Big| \axiom$             \\
$\intgr$      	& $\coloneqq$ & $\Big( \textsf{`0'} \Big| \dots \Big|
                                 \textsf{`9'} \Big)
                                 \Big\{
                                 \textsf{`0'} \Big| \dots \Big| \textsf{`9'}
                                 \Big\}$                                      \\
$\strg$         & $\coloneqq$ & $\textsf{`"'} \Big\{
                                 \textsf{`['} \Big| \textsf{`]'} \Big|
                                 \character \Big\}
                                 \textsf{`"'}$                                \\
$\character$	& $\coloneqq$ & $\textsf{`0'} \Big| \dots \Big| \textsf{`9'}
                                 \Big| \textsf{`A'} \Big| \dots \Big|
                                 \textsf{`Z'} \Big| \textsf{`0'} \Big| \dots
                                 \Big| \textsf{`9'} \Big| \textsf{`+'} \Big|
                                 \textsf{`-'} \Big| \textsf{`/'}
                                 \Big| \textsf{`*'} \Big|\textsf{`\%'} \Big|
                                 \textsf{`\textasciicircum'} \Big| \textsf{`\&'}
                                 \Big| \textsf{`.'} \Big|\textsf{`?'} \Big|
                                 \textsf{`!'} \Big| \textsf{`\_'}$
\end{tabular}

\end{document}
}
    \caption{EBNF for axiom schemata}\label{fig:EBNFaxiom}
\end{figure}

As Figure~\ref{fig:EBNFaxiom} shows, an $\langle \texttt{axiom} \rangle$ is either
of the type \m{ReplaceString} or \m{ReplaceAll}. These are parametrised either
by an $\langle \texttt{item} \rangle$ or a $\langle \texttt{list} \rangle$
of $\langle \texttt{item} \rangle$s. Each $\langle \texttt{item} \rangle$ is
either an $\langle \texttt{int} \rangle$, a $\langle \texttt{string} \rangle$
or again an $\langle \texttt{axiom} \rangle$, so any type of \m{AxiomSchema}.
\\\\
Let us look at an example. The axiom schema \texttt{Induction} can be defined as
follows:
\begin{quote}\vbox%
{\haddockverb\begin{verbatim}
{`[ [[]] [[]] [ [[]] [ [[]] nat] [[]] : [[]] ]:[ [[]] [ [[]] nat]: [[]] ]]`, 0,
  [ 2, {1, 3, `[0[]]`}, 3, 3, 1, {1, 3, {`[1+[[]]]`, 0, [3]}}, 3, 3, 1]}
\end{verbatim}}
\end{quote}

When parsed, this translates to
\begin{minted}{haskell}
induction = ReplaceString ("[ [[]] [[]] [ [[]] [ [[]] nat] [[]] : [[]] ]:" ++
                           "[ [[]] [ [[]] nat]: [[]] ]]") 0
                          [
                          Final 2,
                          ReplaceAll 1 3 (FinalString "[0[]]"),
                          Final 3,
                          Final 3,
                          Final 1,
                          ReplaceAll 1 3
                            (
                            ReplaceString "[1+[[]]]" 0 [Final 3]
                            ),
                          Final 3,
                          Final 3,
                          Final 1
                          ]
\end{minted}

Together with a list of three \emph{external} parameters (note that 3 is the
highest number occurring in the definition of \texttt{Induction}) an axiom can
be built. Consider the following such list:
\begin{minted}{haskell}
["[blabla[n][m][k]]", "[m][k]", "[n]"]
\end{minted}

The integer 0 in the outermost \m{ReplaceString} tells us to replace all
occurrences of \texttt{[[]]} with the \m{SofiaTree}s resulting from recursively
processing the \m{AxiomSchema}ta in the list parametrising \m{ReplaceString}. The
first element in that list is \m{Final 2}, which refers to the second
\emph{external} parameter, namely \texttt{[m][k]}. The second element is of type
\m{ReplaceAll} and tells us that all occurrences of the third \emph{external}
parameter (namely \texttt{[n]}) should be replaced by the result of processing
\m{FinalString "[0[]]"}. That is, all occurences of \texttt{[n]} in
\texttt{[blabla[n][m][k]]} are to be replaced by \texttt{[0[]]}, which results
in \texttt{[blabla[0[]][m][k]]}.
\\\\
The next elements \m{Final 3}, \m{Final 3} and \m{Final 1} refer to the
third, again the third and the first \emph{external} parameter respectively as
before. The sixth element in that list adds another level of recursion. Being of
type \m{ReplaceAll} it tells us that all occurrences of the third
\emph{external} parameter (namely \texttt{[n]}) in the first \emph{external}
parameter (namely \texttt{[blabla[n][m][k]]}) should be replaced with the result
of processing \m{ReplaceString "[1+[[]]]" 0 [Final 3]}. Processing this, we
replace all occurrences of \texttt{[[]]} with the result of processing
\m{Final 3} (i.e.\ \texttt{[n]}) and end up with \texttt{[1+[n]]}. Now we
replace all occurrences of \texttt{[n]} in \texttt{[blabla[n][m][k]]} with
\texttt{[1+[n]]}, which yields \texttt{[blabla[1+[n]][m][k]]}.
\\\\
The last three elements in the outermost list again refer to \emph{external}
parameters respectively. After processing the whole data structure, we end up
with the following \m{Sofia} expression:
\begin{quote}\vbox%
{\haddockverb\begin{verbatim}
[[m][k][blabla[n][m][k]][[n][[n]nat][blabla[n][m][k]]:[blabla[n][m][k]]]:
 [[n][[n]nat]:[blabla[n][m][k]]]] 
\end{verbatim}}
\end{quote}

\input{haddock/WebInterface}

\pagebreak
\begingroup
\setstretch{0.8}
\setlength\bibitemsep{10pt}
\printbibliography\
\endgroup
\end{document}
